\subsubsection{App}
For Into Heart App, we use Android Studio project structure with Gradle project automation tool to do package management. The project bone structure is as followed.

\paragraph{IHApplication.java} is the Application class and holds global variables.
\paragraph{MainActivity.java} is the main activity with a navigation drawer to navigate between fragments, also the inner class EmergencyMonitor monitor heart rate and make emergency call when necessary.
\subparagraph{NavigationDrawerFragment.java} is navigation drawer fragment, let user navigate between fragments.
\subparagraph{DashboardFragment.java} is under MainActivity and let users view the instant heart rate and historical charts, as well as exercise mode.
\subparagraph{AnalysisFragment.java} is under MainActivity and let users view their recent heart rate analysis result  and the diseases they might have.
\subparagraph{SensorsFragment.java} is under MainActivity and let users search the sensors nearby and connect to them.
\subparagraph{RankingFragment.java} is under MainActivity and let user view their score and the ranking among friends. Searching friends, sending friend requests and responding the requests are in this fragment.
\subparagraph{UserInfoFragment.java} is under MainActivity and let user set their basic informations such as name, age, weight, height and emergency telephone.
\subparagraph{LifestyleFragment.java} is under MainActivity and let user rate their lifestyle, including smoking, alcohol, eating disorder, stay up late and overwork.
\paragraph{AddFriendActivity.java} can be entered from RankingFragment, letting user search and send friend request.
\paragraph{RawDataActivity.java} let user view the raw data of data tale "day"
\paragraph{SettingsActivity.java} responds the interaction when user clicks on the setting page.
\paragraph{Splash.java} Splash page.
\paragraph{DiseaseDetailActivity.java} shows the detail of a particular disease using a WebView.
\paragraph{UIComponent/SimpleAlertController.java} is just a wrapper of AlertDialog, for easier usage.
\paragraph{HTTP/API.java} defines the communication prototype between app and server.
\paragraph{HTTP/Connector.java} defines the methods to use API to communicate with server.
\paragraph{HTTP/JCallback.java} is just a wrapper for handler, for easier asynchronous programming especially in networking.
\paragraph{HTTP/Outcome} is the Data model passing in JCallback.
\paragraph{Data/HeartRateContract.java} defines the database model and some simple query/insert methods.
\paragraph{Data/HeartRateStoreController.java} controls all data coming in/out the database.
\paragraph{Data/InstantHeartRateStore.java} stores the recent 60 heart rate data from sensors.
\paragraph{Data/MarkingManager.java} defines the rules of marking scores.
\paragraph{Data/UserStore.java} is the controller to control the user's info and scores, sync the data with server using Connector and sync the data locally using {\bf SharedPreferences}.
\paragraph{BLE/BluetoothLeService.java} is registered service to communicate with Bluetooth LE device.
\paragraph{BLE/GattAttributes.java} stores some constants conforms to Generic Attribute Profile.
\paragraph{BLE/SensorConnectionManager.java} manages the connections between sensor and app, also handle the data sent from sensor and pass to other objects.


\paragraph{res/layout/*} are the layout files defines the static layouts used by activities, fragments, alert dialogs and lists.
\paragraph{res/menu/*} define the action menu bar's items.
\paragraph{res/drawables-*/* and res/minmap-*/*} are image resources.
\paragraph{res/values/*} define the string constants and other static resources.
\paragraph{res/xml/*} define the setting items.





\subsubsection{Server}

Since the stress is not on the server side, only the basic configurations are mentioned below.

For rapidly development, our server-side choice is to use Ubuntu 14.04 server version as OS, with node.js server application, express framework and MongoDB database system. 

\paragraph{routes/users.js} handles all requests with endpoints /users/*.
\paragraph{models/models.js} defines the data models.
\paragraph{app.js} initialize and starts the application, making connection with database.